% !TEX root = ../Projektdokumentation.tex
\section{Einleitung}
\label{sec:Einleitung}

\subsection{Vorstellung der Arsandis GmbH}
\label{sec:Vorstellung_Arsandis}
Die Arsandis GmbH ist ein IT- und Dienstleistungsunternehmen in Angkofen, Pfaffenhofen an der Ilm.
Gegründet wurde die Arsandis im Jahr 2015 und beschäftigt zur Zeit 13 Mitarbeiter.

Das Hauptgeschäftsfeld der Arsandis GmbH liegt in der Beratung von Fertigungsunternehmen im Bereich des Product Lifecycle Managements (PLM).
Darüber hinaus hat das Unternehmen ein starkes Interesse an zukunftsorientierten Technologien und verfolgt aktiv die Entwicklungen in Bereichen wie Augmented Reality, Virtual Reality und Internet of Things.
Diese Technologien bieten innovative Lösungen für die Industrie der Zukunft.
Weitere zukunftsorientierte Technologien werden ebenfalls verfolgt, darunter zum Beispiel AR-, VR- und IoT-basierte Lösungen für die Industrie der Zukunft.
Die Arsandis GmbH hat ein breites Spektrum an Kunden aus verschiedenen Branchen.
Dazu gehören Unternehmen aus dem Maschinenbau, der Automobil- und Luftfahrtindustrie sowie der HighTech- und Telekommunikationsbranche.

Als offizieller Partner von PTC liegt das Hauptaugenmerk hier meist auf dem PTC-Portfolio, das ein umfangreiches Ökosystem für die Fertigungsindustrie bereitstellt.

\subsection{Projektumfeld}
\label{sec:Projektumfeld}
Das Projekt wird im Auftrag der amerikanischen Firma \ac{AWS} durchgeführt, die in der Herstellung von Wasserpumpen tätig ist.
\ac{AWS} ist ein Tochterunternehmen von Xylem, einem Unternehmen, das sich auf die Entwicklung und Bereitstellung von Wassertechnologien spezialisiert hat.
\ac{AWS} nutzt den Windchill Server von Xylem, der als \ac{XGV} bekannt ist.
Die meisten Tochterunternehmen von Xylem, einschließlich \ac{AWS}, sind auf diesem Server als Organisation eingerichtet.

Durchgeführt wird das Projekt von mir in den Räumen der Arsandis GmbH.
Im Vornherein wurde für dieses Projekt eine Windchill Entwicklungsumgebung auf unserem Unternehmensserver eingerichtet.
Auf dieser findet jegliche Entwicklungsarbeit statt.
Nach Abschluss der Implementierungsphase sollen die Softwareanpassungen in das Testsystem von Xylem integriert werden.
Die finalen Tests und das Deployment auf das Produktivsystem werden von Mitarbeitern von AWS durchgeführt.

\subsection{Projektziel} 
\label{sec:Projektziel}
Das Projekt verfolgt zwei Ziele.
Zum einen soll das bestehende \ac{ECM} von \ac{AWS} angepasst und erweitert werden.
Dem \ac{ECM}-Prozess sollen neue Funktionen hinzugefügt werden, die die Robustheit des Prozesses erhöhen und die benötigte Durchlaufzeit verringern.

Andererseits sollen neue Organisationen in das Windchill-\ac{ECM} eingebunden werden.
Dafür soll der manuelle Prozess der Befüllung von Excel-Dateien durch Windchill ersetzt werden.
User sollen den \ac{ECM}-Prozess in Windchill initiieren, wodurch alle Daten, die für das \ac{ECM} relevant sind, gebündelt an einem Ort abgelegt werden.
Da sich die CAD-Dateien ohnehin schon auf dem Windchill-Server befinden, lassen sich CAD-Dateien und \ac{ECM} somit praktisch verknüpfen.

\subsection{Projektbegründung} 
\label{sec:Projektbegruendung}
Es wird eine erhöhte Robustheit des aktuellen Prozesses angestrebt, indem das manuelle Erstellen von \acp{PR} automatisiert wird, was auch zu einer Zeitersparnis führt.
Außerdem soll der \ac{ECM}-Prozess in Organisationen eingeführt werden, die bisher entweder auf Excel-Tabellen und selbst entwickelte Webanwendungen oder auf persönliche Besprechung mit dem Produktmanager zurückgegriffen haben, um diese Prozesse abzubilden.
Dies führt zum einen zu einer erheblichen Zeitersparnis, da viele manuelle Vorgänge, wie das Befüllen von einer Excel-Tabelle, wegfallen.
Dadurch wird wiederum auch die Robustheit deutlich erhöht, da repetitive Tätigkeiten nicht mehr vergessen werden können und häufige Fehler vermieden werden können.

\subsection{Projektabgrenzung}
Das Abschlussprojekt nimmt ungefähr 80 \% des kompletten Projekts ein.
Die restlichen 20 \% werden nach Abschluss des Abschlussprojekts angegangen.
Dann ist geplant, noch einige weitere Quality-Of-Life Verbesserungen zu implementieren.

\subsection{Projektschnittstellen}
\label{sec:Projektschnittstellen}

\subsubsection{Windchill}
Windchill ist eine \ac{PLM} Webanwendung von PTC, die das Verwalten von Unternehmensobjekten für die Herstellungsindustrie vereinfacht.
Besonders Wert wird dabei auf die Verwaltung von CAD-Modellen gelegt.
CAD-Modelle sind am Computer gefertigte zwei- und dreidimensionale Zeichnungen von Konstruktionsobjekten.
Mit dem Windchill Workgroup Manager existiert eine Schnittstelle zwischen Windchill und vielen CAD-Anwendungen, die auf dem Markt existieren (wie \zB Creo Parametric, CATIA oder Solidworks).

\subsubsection{\acs{XGV}}
Der \acl{XGV} ist der Windchill-Server von Xylem.
Die meisten Tochterfirmen von Xylem sind auf diesem Server als eigenständige Organisation eingerichtet.
Für die Durchführung des Projekts hat dieser Server allerdings keine Relevanz, da die endgültige Übernahme auf das Produktivsystem nicht von uns durchgeführt wird.
Dennoch wurde er aus Gründen der Vollständigkeit aufgeführt.

\subsubsection{\ac{XGV}-Trainingsserver}
Dieses Testsystem ist ein Klon vom \ac{XGV}-Produktivserver.
Hier nimmt der Kunde die betriebseigenen Tests vor, um unsere Anpassungen zu überprüfen.
Zusammen mit meinem Projektleiter werde ich die Übernahme von unserem Entwicklungssystem auf den Trainingsserver durchführen.

\subsubsection{\acs{AWS}}
Auch \ac{AWS} hat eigene Tochterunternehmen, die in Windchill als Organisation eingerichtet sind.
Darunter sind zum Beispiel FHD, Vogel oder Lowara.
Wichtig ist, dass alle Änderungen, die von uns vorgenommen werden, ausschließlich die Organisationen von \ac{AWS} betreffen.

\subsubsection{Workflows}
Workflows sind Objekte in Windchill, mit denen man Unternehmensprozesse abbilden kann.
Dafür stellt Windchill einen Workflow-Editor bereit, mit dem man die Unternehmensprozesse in einer praktischen Benutzeroberfläche definieren kann.
Innerhalb von Workflows lässt sich auch direkt Code definieren, der dann mit dem Workflow ausgeführt wird.
Hier kann man auch auf Klassen und Methoden innerhalb der Windchill-Codebase verweisen.
Die folgenden Konzepte wurden und werden mit Hilfe von Workflows umgesetzt.

\subsubsection{\acl{ECM}}
Das Änderungsmanagement ist ein essenzieller Bestandteil von Windchill.
Teile, wie \zB CAD-Dateien oder Dokumente kommen in ihrem Lebenszyklus zumeist irgendwann an den Punkt, an dem sie für die Produktion freigegeben werden.
Da bereits umfassende Ressourcen für die Produktion aufgewendet wurden, muss eine nachträgliche Änderung gut begründet werden.
Schließlich wurden bereits Maschinen/Maschinenteile bestellt, eingerichtet oder modifiziert, um das Produkt herzustellen.
Demzufolge müssen nachträgliche Änderungen nun in Absprache mit verschiedenen Unternehmensabteilungen koordiniert werden.
Der Change Admin ist für die Koordination des Change Managements verantwortlich.
In Windchill gibt es eigentlich drei verschiedene Change Administrator Rollen, vereinfacht wird folgend aber lediglich vom Change Admin gesprochen.

Hier kommt das Änderungsmanagement ins Spiel.
Es umfasst insgesamt drei Bestandteile, die alle in Windchill implementiert sind: \acl{ECR}, \acl{ECN} und \acl{ECA}.
Diese werden in den nächsten Abschnitten genauer erläutert.

\subsubsection{\acl{ECR}}
Der Änderungsantrag ist der erste Schritt im Änderungsmanagement.
Grob gesagt wird im \ac{ECR} geklärt, \underline{ob} die Änderung umgesetzt werden kann.
Der Change Admin wählt die Unternehmensabteilungen aus, die über die Durchführung der Änderung abstimmen sollen.
Die Abteilungen stimmen dann entweder für oder gegen die geplante Änderung.
Nur wenn alle befragten Abteilungen für die Änderung sind, wird der Prozess mit der \ac{ECN} fortgesetzt.
Stimmt mindestens eine Abteilung dagegen, so wird der Änderungsantrag verworfen.

\subsubsection{\acl{ECN}}
Die Änderungsmitteilung ist der zweite Schritt im Änderungsmanagement.
Hier geht es darum, dass festgelegt wird, \underline{wie} die Änderung umgesetzt wird.
Dafür legt der Change-Admin zuerst fest, welche Abteilungen an der Umsetzung der Änderungen beteiligt sein sollen.
Danach bestimmt er, welche Nutzer aus der Abteilung an der Umsetzung der Änderung arbeiten sollen.

\subsubsection{\acl{ECA}}
Die Änderungsaufgabe ist der dritte und letzte Schritt im Änderungsmanagement.
Die geplanten Änderungen werden hier umgesetzt und anschließend vom Change-Admin überprüft.
Ist der Change-Admin unzufrieden mit den umgesetzten Änderungen, so kann er eine Überarbeitung von den relevanten Abteilungen anfordern.
Sobald der Change-Administrator seine Zustimmung zu den Änderungen erteilt hat, markiert dies den erfolgreichen Abschluss der Änderungsaufgabe und damit auch des gesamten Change-Prozesses.

\subsubsection{\acl{PR}}
Der Erhöhungsantrag ist nicht direkt Teil dieses Projekts.
Allerdings werden neue Mechanismen im \ac{ECM}-Workflow eingeführt, die das manuelle Erstellen von Erhöhungsanträgen ersetzen soll.
Demzufolge wird der Erhöhungsantrag kurz erläutert.

Ein Windchill-Objekt durchläuft stets verschiedene Phasen seines Lebenszyklus.
Im Moment der Erstellung befindet es sich üblicherweise im Status `In Arbeit`.
Um das Objekt in einen anderen Zustand zu überführen, wie beispielsweise `Freigegeben`, wird ein Erhöhungsantrag durchlaufen.
In diesem Prozess entscheidet der Genehmiger, ob eine Änderung des Status des Objekts angebracht ist.