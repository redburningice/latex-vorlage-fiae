% !TEX root = ../Projektdokumentation.tex
\section{Einleitung}
\label{sec:Einleitung}

\subsection{Vorstellung der Arsandis GmbH}
\label{sec:Vorstellung_Arsandis}
Die Arsandis GmbH ist ein IT- und Dienstleistungsunternehmen in Angkofen, Pfaffenhofen an der Ilm.
Gegründet wurde die Arsandis im Jahr 2015 und beschäftigt zur Zeit 13 Mitarbeiter.

Das Hauptgeschäftsfeld der Arsandis GmbH liegt in der Beratung von Fertigungsunternehmen im Bereich des Product Lifecycle Managements (PLM).
Darüber hinaus hat das Unternehmen ein starkes Interesse an zukunftsorientierten Technologien und verfolgt aktiv die Entwicklungen in Bereichen wie Augmented Reality, Virtual Reality und Internet of Things.
Diese Technologien bieten innovative Lösungen für die Industrie der Zukunft.
Weitere zukunftsorientierte Technologien werden ebenfalls verfolgt, darunter zum Beispiel AR-, VR- und IoT-basierte Lösungen für die Industrie der Zukunft.
Die Arsandis GmbH hat ein breites Spektrum an Kunden aus verschiedenen Branchen.
Dazu gehören Unternehmen aus dem Maschinenbau, der Automobil- und Luftfahrtindustrie sowie der HighTech- und Telekommunikationsbranche.

Als offizieller Partner von PTC liegt das Hauptaugenmerk hier meist auf dem PTC-Portfolio, das ein umfangreiches Ökosystem für die Fertigungsindustrie bereitstellt.

\subsection{Projektumfeld}
\label{sec:Projektumfeld}
Durchgeführt wird das Projekt von mir in den Räumen der Arsandis GmbH, bei der ich meine Ausbildung absolviere.
Das Projekt wird im Auftrag der amerikanischen Firma \ac{AWS} durchgeführt, die in der Herstellung von Wasserpumpen tätig ist.
\ac{AWS} ist ein Tochterunternehmen von Xylem, einem Unternehmen, das sich auf die Entwicklung und Bereitstellung von Wassertechnologien spezialisiert hat.

\subsubsection{Systeme}

\subsubsection*{Produktivsystem}
Der \acl{XGV} ist der Windchill-Server von Xylem und wird von PTC in einer Cloud gehostet.
Der Login auf dieses System erfolgt für die User über Single-Sign-On.
Die meisten Tochterfirmen von Xylem sind auf diesem Server als eigenständige Organisation eingerichtet.
Für die Durchführung des Projekts hat dieser Server allerdings keine Relevanz, da die endgültige Übernahme auf das Produktivsystem nicht von uns durchgeführt wird.
Wichtig ist, dass alle Änderungen, die von uns vorgenommen werden, ausschließlich die Organisationen von \ac{AWS} betreffen.

\subsubsection*{Testsystem}
Dieses Testsystem ist ein Klon vom \ac{XGV}-Produktivserver und wird von Xylem gehostet.
Der Login auf dieses System erfolgt für die User ebenfalls über Single-Sign-On.
Hier nimmt der Kunde die betriebseigenen Tests vor, um unsere Anpassungen zu überprüfen.
Zusammen mit meinem Projektleiter werde ich die Übernahme von unserem Entwicklungssystem auf den Trainingsserver durchführen.

\subsubsection*{Entwicklungssystem}
Das Entwicklungssystem wird von Arsandis gehostet und verwaltet (Hyper-V VM) und wurde im Vorfeld als Klon des Testsystems für dieses Projekts eingerichtet, damit sämtliche Entwicklungsarbeiten darin vorgenommen werden können.
Statt Single-Sign-On wurden hier lokale Testuser auf einem lokalen LDAP eingerichtet.
Auch das Arsandis-Interne Testing wird hier stattfinden.
Die Abschlussprojektarbeit wird - bis auf des Deployment auf den Trainingsserver - vollständig auf diesem Entwicklungssystem stattfinden.

\subsection{Projektziel} 
\label{sec:Projektziel}
Das bestehende \ac{ECM} von \ac{AWS} soll angepasst und erweitert werden.
Dem \ac{ECM}-Prozess sollen neue Funktionen hinzugefügt werden, die die Robustheit des Prozesses erhöhen und die benötigte Durchlaufzeit verringern.

%Andererseits sollen neue Organisationen in das Windchill-\ac{ECM} eingebunden werden.
%Dafür soll der manuelle Prozess der Befüllung von Excel-Dateien durch Windchill ersetzt werden.
%User sollen den \ac{ECM}-Prozess in Windchill initiieren, wodurch alle Daten, die für das \ac{ECM} relevant sind, gebündelt an einem Ort abgelegt werden.
%Da sich die CAD-Dateien ohnehin schon auf dem Windchill-Server befinden, lassen sich CAD-Dateien und \ac{ECM} somit praktisch verknüpfen.

\subsection{Projektbegründung} 
\label{sec:Projektbegruendung}
Es wird eine erhöhte Robustheit des aktuellen Prozesses angestrebt, indem das manuelle Erstellen von \acp{PR} automatisiert wird, was auch zu einer Zeitersparnis führt.
%Außerdem soll der \ac{ECM}-Prozess in Organisationen eingeführt werden, die bisher entweder auf Excel-Tabellen und selbst entwickelte Webanwendungen oder auf persönliche Besprechung mit dem Produktmanager zurückgegriffen haben, um diese Prozesse abzubilden.
%Dies führt zum einen zu einer erheblichen Zeitersparnis, da viele manuelle Vorgänge, wie das Befüllen von einer Excel-Tabelle, wegfallen.
%Dadurch wird wiederum auch die Robustheit deutlich erhöht, da repetitive Tätigkeiten nicht mehr vergessen werden können und häufige Fehler vermieden werden können.

\subsection{Projektabgrenzung}
Das Abschlussprojekt nimmt ungefähr 80 \% des kompletten Projekts ein.
Die restlichen 20 \% werden nach Fertigstellung des Abschlussprojekts angegangen.
Es ist geplant, noch einige weitere Quality-Of-Life Verbesserungen zu implementieren.

\subsection{Projektschnittstellen}
\label{sec:Projektschnittstellen}

Das System, an dem das Projekt durchgeführt wird, interagiert mit einigen anderen System, wie zum Beispiel
\begin{itemize}
    \item CAD-Anwendungen über den Windchill Workgroup Manager
    \item Microsoft Sharepoint zur Datenübertragung
\end{itemize}

Diese sind allerdings für das Projekt nicht weiter von Belang und wurden hier deshalb nur zur Vollständigkeit aufgeführt.

Bei den Nutzern, die mit dem System interagieren, handelt es sich vor allem um Ingenieure und Manager.
Ingenieure nutzen das System hauptsächlich dafür, CAD-Modelle zu erstellen, zu ändern und zu verwalten.
Manager hingegen genehmigen und koordinieren Unternehmensprozesse, wie \zB das Änderungsmanagement, auf das im Kapitel \ref{sec:ECM} noch näher eingegangen wird.
Da der Windchill-Server schon lange im Einsatz ist, sind alle Mitarbeiter mit dem System bestens vertraut.

\subsection{Vorraussetzungen für das Verständnis}

Windchill bietet umfassende Funktionen zur Verwaltung von Unternehmensdaten.
Im Folgenden sollen einige dieser Funktionen vorgestellt werden, damit die Inhalte des Abschlussprojekts nachvollziehbar sind.

\subsubsection{Workflows}
Workflows sind Objekte, mit denen man Unternehmensprozesse abbildet.
Dafür stellt Windchill einen Workflow-Editor bereit, mit dem man die Unternehmensprozesse in einer praktischen Benutzeroberfläche definieren kann.
In Workflows lässt sich auch direkt Code definieren, der dann zusammen mit dem Workflow ausgeführt wird.
Hier kann man auch auf Klassen und Methoden innerhalb der Windchill-Codebasis verweisen.
Die folgenden Konzepte wurden und werden mit Hilfe von Workflows umgesetzt.

\subsubsection{\acl{ECM}}
\label{sec:ECM}
Das Änderungsmanagement ist ein essenzieller Bestandteil von Windchill.
Damit werden \glqq Änderungen eines Produkts vollständig definiert und kontrolliert.
Dies sorgt dafür, dass Aufgaben an die Verantwortlichen in einem wiederholbaren und automatisiertem Workflow gesendet werden \grqq{} \cite{WindchillChangeManagement2}.
Der Change Admin ist für die Koordination des Change Managements verantwortlich.
In Windchill gibt es eigentlich drei verschiedene Change Administrator Rollen, vereinfacht wird in diesem Projekt aber lediglich vom Change Admin gesprochen.

In Windchill sind drei Facetten des Änderungsmanagements implementiert: \acl{ECR}, \acl{ECN} und \acl{ECA}.
Diese werden in den nächsten Abschnitten genauer erläutert.

\subsubsection{\acl{ECR}}
Der Änderungsantrag ist der erste Schritt im Änderungsmanagement.
Grob gesagt wird im \ac{ECR} geklärt, \underline{ob} die Änderung umgesetzt werden kann.
Der Change Admin wählt Unternehmensabteilungen aus.
Diese stimmen dann darüber ab, ob die Änderung vollzogen werden soll.
Nur wenn alle befragten Abteilungen für die Änderung sind, wird der Prozess mit der \ac{ECN} fortgesetzt.
Stimmt mindestens eine Abteilung dagegen, so wird der Änderungsantrag verworfen.

\subsubsection{\acl{ECN}}
Die Änderungsmitteilung ist der zweite Schritt im Änderungsmanagement.
Hier geht es darum, dass festgelegt wird, \underline{wie} die Änderung umgesetzt wird.
Dafür legt der Change Admin die Abteilungen fest, die die Änderungen durchführen sollen.
Außerdem werden in diesem Schritt die Aufgaben definiert, die umgesetzt werden müssen, um die Änderung zu vollziehen.

\subsubsection{\acl{ECA}}
Die Änderungsaufgabe ist der dritte und letzte Schritt im Änderungsmanagement.
Die geplanten Änderungen werden hier umgesetzt und anschließend vom Change-Admin überprüft.
Ist der Change-Admin unzufrieden mit den umgesetzten Änderungen, so kann er eine Überarbeitung von den relevanten Abteilungen anfordern.
Sobald der Change-Administrator seine Zustimmung zu den Änderungen erteilt hat, markiert dies den erfolgreichen Abschluss der Änderungsaufgabe und damit auch des gesamten Change-Prozesses.

\subsubsection{\acl{PR}}
Der Erhöhungsantrag ist nicht direkt Teil dieses Projekts.
Allerdings werden neue Mechanismen im \ac{ECM}-Workflow eingeführt, die das manuelle Erstellen von Erhöhungsanträgen ersetzen soll.
Demzufolge wird der Erhöhungsantrag kurz erläutert.

Ein Windchill-Objekt durchläuft stets verschiedene Phasen seines Lebenszyklus.
Im Moment der Erstellung befindet es sich üblicherweise im Status `In Arbeit`.
Um das Objekt in einen anderen Zustand zu überführen, wie beispielsweise `Freigegeben`, wird ein Erhöhungsantrag durchlaufen, wo ein Genehmiger entscheided, ob eine Änderung des Status des Objekts angebracht ist.