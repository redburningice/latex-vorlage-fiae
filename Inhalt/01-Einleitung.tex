% !TEX root = ../Projektdokumentation.tex
\section{Einleitung}
\label{sec:Einleitung}

\subsection{Vorstellung der Arsandis GmbH}
\label{sec:Vorstellung_Arsandis}
Die Arsandis GmbH ist ein IT- und Dienstleistungsunternehmen in Angkofen, Pfaffenhofen an der Ilm.
Gegründet wurde die Arsandis im Jahr 2015 und beschäftigt zur Zeit 13 Mitarbeiter.

Das Hauptgeschäftsfeld der Arsandis GmbH liegt in der Beratung von Fertigungsunternehmen im Bereich des Product Lifecycle Managements (PLM).
Darüber hinaus hat das Unternehmen ein starkes Interesse an zukunftsorientierten Technologien und verfolgt aktiv die Entwicklungen in Bereichen wie Augmented Reality, Virtual Reality und Internet of Things.
Diese Technologien bieten innovative Lösungen für die Industrie der Zukunft.Weitere zukunftsorientierte Technologien werden ebenfalls verfolgt, darunter zum Beispiel AR-, VR- und IoT-basierte Lösungen für die Industrie der Zukunft.Die Arsandis GmbH hat ein breites Spektrum an Kunden aus verschiedenen Branchen.
Dazu gehören Unternehmen aus dem Maschinenbau, der Automobil- und Luftfahrtindustrie sowie der HighTech- und Telekommunikationsbranche.

Als offizieller Partner von PTC liegt das Hauptaugenmerk hier meist auf dem PTC-Portfolio, das ein umfangreiches Ökosystem für die Fertigungsindustrie bereitstellt.

\subsection{Projektbeschreibung}
\label{subsec:Projektbeschreibung}

\subsection{Projektumfeld}
\label{sec:Projektumfeld}
Das Projekt wurde im Auftrag der amerikanischen Firma \ac{AWS} durchgeführt, die in der Herstellung von Wasserpumpen tätig ist.
\ac{AWS} ist ein Tochterunternehmen von Xylem, einem Unternehmen, das sich auf die Entwicklung und Bereitstellung von Wassertechnologien spezialisiert hat.
\ac{AWS} nutzt den Windchill Server von Xylem, der als \ac{XGV} bekannt ist.
Die meisten Tochterunternehmen von Xylem, einschließlich \ac{AWS}, sind auf diesem Server als Organisation eingerichtet.

Durchgeführt wurde das Projekt von mir in den Räumen von der Arsandis GmbH.
Im Rahmen des Projekts wurde eine Windchill Entwicklungsumgebung auf unserem Unternehmensserver eingerichtet.
Nach Abschluss der Implementierungsphase sollen die Softwareanpassungen in das Testsystem von Xylem integriert werden.

\subsection{Projektziel} 
\label{sec:Projektziel}
Ziel des Projektes ist es, das bestehende \ac{ECM} von \ac{AWS} anzupassen und zu erweitern.
Dabei sollen die folgenden Funktionen implementiert werden:
\begin{itemize}
	\item Übersichtlichere und besser nachvollziehbarere Darstellung der Workflowprozesse, die es einfacher macht, den Workflow in Zukunft zu erweitern.
	\item Hinzufügen neuer Abteilungen zum Prozess
	\item Die \ac{JSP}-Seite zur Auswahl von Rollen soll durch eine \ac{OOTB} Lösung ersetzt werden.
	Dies gewährleistet, dass zukünftige Upgrades des Windchillsystems zu keinen Kompatibilitätsproblemen führen.
	\item Die Implementierungsaufgaben der Abteilungen aus der \ac{ECN} sollen in eine \ac{ECA} ausgelagert werden, um den Change Prozess nachvollziehbarer zu gestalten.
	Dies erlaubt es uns auch zusätzliche \ac{ECA} spezifische Windchill Features zu nutzen.
	\item Die Logik der Startparameter soll angepasst werden, dazu muss Code im Workflow angepasst werden.
	\item Objekte, die gerade vom Change Admin überprüft werden, sollen für eine weitere Bearbeitung per Programmierung gesperrt werden, damit keine ungewünschten Änderungen erfolgen können.
	\item Workflowprozesse so anpassen, dass auch Windchill Admins ohne Zugriff zur Virtuellen Maschine des Produktivsystems Änderungen am Prozess durchführen können.
	\item Umbenennen von Rollen, die am Prozess beteiligt sind.
\end{itemize}

\subsection{Projektbegründung} 
\label{sec:Projektbegruendung}
\begin{itemize}
	\item Warum ist das Projekt sinnvoll (\zB Kosten- oder Zeitersparnis, weniger Fehler)?
	\item Was ist die Motivation hinter dem Projekt?
\end{itemize}


\subsection{Projektschnittstellen} 
\label{sec:Projektschnittstellen}
\begin{itemize}
	\item Mit welchen anderen Systemen interagiert die Anwendung (technische Schnittstellen)?
	\item Wer genehmigt das Projekt \bzw stellt Mittel zur Verfügung? 
	\item Wer sind die Benutzer der Anwendung?
	\item Wem muss das Ergebnis präsentiert werden?
\end{itemize}


\subsection{Projektabgrenzung} 
\label{sec:Projektabgrenzung}
\begin{itemize}
	\item Was ist explizit nicht Teil des Projekts (\insb bei Teilprojekten)?
\end{itemize}

\subsection{Voraussetzungen für das Verständnis}
\label{subsec:VerstaendnisVoraussetzungen}
\subsubsection{\acl{ECM}}
Das Änderungsmanagement ist ein essenzieller Bestandteil von Windchill.
Teile, wie z.\ B. CAD-Dateien oder Dokumente kommen in ihrem Lebenszyklus zumeist irgendwann an den Punkt, an dem sie für die Produktion freigegeben werden.
Da bereits umfassende Ressourcen für die Produktion aufgewendet wurden, muss eine nachträgliche Änderung gut begründet werden.
Schließlich wurden bereits Maschinen/Maschinenteile bestellt, eingerichtet oder modifiziert, um das Produkt herzustellen.
Demzufolge müssen nachträgliche Änderungen nun in Absprache mit verschiedenen Unternehmensabteilungen koordiniert werden.

Hier kommt das Änderungsmanagement ins Spiel.
Es umfasst insgesamt drei Bestandteile: \acl{ECR}, \acl{ECN} und \acl{ECA}.
Diese sollen in den nächsten Abschnitten genauer erläutert werden.

\subsubsection{\acl{ECR}}


\subsubsection{\acl{ECN}}


\subsubsection{\acl{ECA}}

