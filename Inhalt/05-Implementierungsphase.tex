% !TEX root = ../Projektdokumentation.tex
\section{Implementierungsphase}  %todo: finish section
\label{sec:Implementierungsphase}

\subsubsection{Anpassen der Workflow-Templates}
\subsubsection*{Umstrukturierung der Nodes zur besseren Übersichtlichkeit}
Exemplarisch wird hier nur die neue Struktur des ECRs gezeigt, da sich hier die größten Änderungen ergeben haben.
Der ECN Workflow hat allerdings auch eine Umstrukturierung erfahren.
Es wurde besonders Wert darauf gelegt, dass die Elemente in einer klaren Struktur angeordnet werden und sich Linien möglichst nicht kreuzen.
Man kann deutlich erkennen, dass der Workflow nun deutlich aufgeräumter und strukturierter wirkt.
Dadurch wird vor allem das Leseverständnis erhöht.
Zudem wurde der große Block im unteren Teil von \autoref{WorkflowVergleich1} mit einem seperaten Block (Department Approval Process) ersetzt.

Ein vorher-nachher Vergleich findet sich im \Anhang{WorkflowVergleich}.



\subsubsection{Ergänzung von Lebenszyklusstatus im ECA}
\label{LebenszyklusstatusECA}

\subsubsection{Erstellen neuer Types}
Damit der Workflow für ECA-E und ECA-NE auch im System getriggert wird, müssen 

\begin{itemize}
	\item Anlegen neuer Types
	\subitem OIR
	\subitem Lifecycle Template
	\item Anpassen der Lifecycle Templates
	\item Erstellung der Java-Klassen
\end{itemize}

