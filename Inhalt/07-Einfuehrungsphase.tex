% !TEX root = ../Projektdokumentation.tex
\section{Einführungsphase}
\label{sec:Einfuehrungsphase}

Das Deployment findet zusammen mit meinem Projektleiter auf einem Terminal-Only Linux-Server statt.
Die Übertragung der Änderungen auf den Testserver läuft komplett manuell ab.
Mithilfe des Command-Line (CLI) LoadFile Tools von PTC lassen sich viele Windchill Daten importieren.
Auf diese Weise konnte ich die Types auf dem Server importieren.
Der kompilierte Java Code und die Datei für die Rollendefinition wurde in der Windchill Codebase abgelegt.

Der Import der Workflows, Life Cycle Templates und OIRs funktioniert über das UI -- ebenfalls manuell -- direkt im Windchill Server.

Die Rollen wurden für zwei Team Templates händisch ausgetauscht.
Ursprünglich war für das Abschlussprojekt vorgesehen, dass dies mit einem Java Programm automatisiert für alle Team Templates der Organisation funktioniert.
Durch den Zeitdruck wurde dieses Vorhaben allerdings nach hinten verschoben und zudem auch an einen anderen Mitarbeiter übergeben.

Die Übertragung auf den Produktivserver, die nicht von uns durchgeführt wird, wird automatisiert ablaufen.
Dafür bereitet ein Experte von AWS ein ANT Buildscript vor.

