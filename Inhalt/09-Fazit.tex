% !TEX root = ../Projektdokumentation.tex
\section{Fazit} 
\label{sec:Fazit}

\subsection{Soll-/Ist-Vergleich}
\label{sec:SollIstVergleich}

Das Projektziel wurde zur Deadline nicht eingehalten.
Grund dafür ist, dass sich auf der einen Seite die Anforderungen im Vergleich zum Start geändert haben.
Auf der anderen Seite war die Zeit von rund 80 Stunden deutlich zu knapp bemessen für die Fertigstellung des Projekts.
Darum wurde die Deadline um einige Wochen nach hinten verschoben.
Das hat auch zur Folge, dass das Projektbudget aufgestockt werden musste.

Durch den Zeitdruck und eine suboptimale Kommunikation sind außerdem einige Fehler in der Implementierung passiert.
Teilweise wurden Anforderungen nicht optimal umgesetzt oder anders umgesetzt, als sich der Kunde das vorgestellt hat.
Die Fehler wurden von den Testern entdeckt und dokumentiert.
Das Projekt wird auch über das Abschlussprojekt hinaus forgesetzt, dabei sollen die Fehler ausgebessert werden.
Trotzdem ist der Auftraggeber grundsätzlich zufrieden mit dem Ergebnis, denn es soll noch ein weiteres Projekt folgen.

Auch das programmatische Umbenennen von Rollen wurde in der geplanten Zeit nicht geschafft.
Deswegen wurde diese Aufgabe an einen anderen Mitarbeiter der Arsandis ausgelagert.
Dies ist aber nicht weiter tragisch, da diese Funktion erst für den Umzug auf den Produktivserver wirklich relevant ist.
Für das Testen auf den Testservern wurden einige Team-Templates händisch angepasst, wodurch diese Phase des Projekts nicht beeinflusst wurde.

Im Projektantrag wurde bei den Aufgaben die Erstellung eines Use-Case Diagramms erwähnt, dies wurde während des Projekts allerdings fallen gelassen, da es für das Projekt nicht notwendig erschien.

Abschließend ist der Zeitaufwand für das Anfertigen der Dokumentation unerwartet hoch ausgefallen, dafür wurde weniger Zeit in das Erstellen der Aktivitätsdiagramme investiert, damit 80 Stunden nicht überschritten werden.

Wie in Tabelle~\ref{tab:Vergleich} zu erkennen ist, konnte die Zeitplanung nicht wie erwartet eingehalten werden.
\tabelle{Soll-/Ist-Vergleich}{tab:Vergleich}{Zeitnachher.tex}


\subsection{Lessons Learned}
\label{sec:LessonsLearned}

Dieses Projekt war eine hervorragende Möglichkeit, das bis dahin in der Ausbildung Gelernte anzuwenden.
Da dies das erste große Projekt war, in dem ich mich mit der Zeitplanung hauptsächlich selbst befassen sollte, war dies zu Beginn eine große Herausforderung.
Allerdings waren die Schulstunden zum Thema Projektmanagement eine große Hilfe, um Kundenanforderungen richtig zu planen.

Zudem habe ich durch den regelmäßigen Austausch mit dem Kunden viel im Bereich der Kommunikation dazugewonnen.
Es wurden zwar bereits vor dem Abschlussprojekt regelmäßige Gespräche mit langjährigen Kunden geführt, die Kunden dieses Projekts kannten wir allerdings noch nicht zuvor.
Teilweise gab es Missverständnisse, die nach einem klärenden Gespräch beseitigt wurden.
Dies war eine wertvolle Erfahrung, die im zukünftigen Berufsleben Gold wert ist.
