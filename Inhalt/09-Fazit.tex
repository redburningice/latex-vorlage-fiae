% !TEX root = ../Projektdokumentation.tex
\section{Fazit} 
\label{sec:Fazit}

\subsection{Soll-/Ist-Vergleich}
\label{sec:SollIstVergleich}

\begin{itemize}
	\item Wurde das Projektziel erreicht und wenn nein, warum nicht?
	\item Ist der Auftraggeber mit dem Projektergebnis zufrieden und wenn nein, warum nicht?
	\item Wurde die Projektplanung (Zeit, Kosten, Personal, Sachmittel) eingehalten oder haben sich Abweichungen ergeben und wenn ja, warum?
	\item Hinweis: Die Projektplanung muss nicht strikt eingehalten werden. Vielmehr sind Abweichungen sogar als normal anzusehen. Sie müssen nur vernünftig begründet werden (\zB durch Änderungen an den Anforderungen, unter-/überschätzter Aufwand).
\end{itemize}

\paragraph{Beispiel (verkürzt)}
Wie in Tabelle~\ref{tab:Vergleich} zu erkennen ist, konnte die Zeitplanung bis auf wenige Ausnahmen eingehalten werden.
\tabelle{Soll-/Ist-Vergleich}{tab:Vergleich}{Zeitnachher.tex}


\subsection{Lessons Learned}
\label{sec:LessonsLearned}

Dieses Projekt war eine hervorragende Möglichkeit, das bis dahin in der Ausbildung Gelernte anzuwenden.
Da dies das erste große Projekt war, in dem der Autor sich mit der Zeitplanung hauptsächlich selbst befassen sollte, war dies zu Beginn eine große Herausforderung.
Allerdings waren die Schulstunden zum Thema Projektmanagement eine große Hilfe, um Kundenanforderungen richtig zu planen.

Zudem hat der Prüfling durch den regelmäßigen Austausch mit dem Kunden viel im Bereich der Kommunikation dazugewonnen.
Es wurden zwar bereits vor dem Abschlussprojekt Gespräche mit langjährigen Kunden geführt, die Kunden dieses Projekts kannten wir allerdings noch nicht zuvor.
Teilweise gab es Missverständnisse, die nach einem klärenden Gespräch beseitigt wurden.
Dies war eine wertvolle Erfahrung, die im zukünftigen Berufsleben Gold wert ist.

\subsection{Ausblick}
\label{sec:Ausblick}

\begin{itemize}
	\item Wie wird sich das Projekt in Zukunft weiterentwickeln (\zB geplante Erweiterungen)?
\end{itemize}
