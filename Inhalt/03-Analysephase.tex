% !TEX root = ../Projektdokumentation.tex
\section{Analysephase} 
\label{sec:Analysephase}


\subsection{Ist-Analyse} 
\label{sec:IstAnalyse}
Dieser Abschnitt erhält Auszüge aus dem Projektantrag, um eine Übersicht über die aktuelle Situation des Systems zu geben.

Der \ac{ECM} besteht aus den zwei Komponenten \ac{ECR} und \ac{ECN}.
Aktuell werden Aufgaben im \ac{ECN} getrackt, dies ist so von Windchill eigentlich nicht vorgesehen.
In Zukunft sollen Aufgaben deswegen in einem \ac{ECA} abgebildet werden.
Dies führt dazu, dass die konkreten Implementierungsaufgaben logisch von der Planungsphase (ECN) abgegrenzt wird, was es dem Change Admin einfacher macht, einen Überblick über die aktuelle Situation zu bekommen.

Im \ac{ECM} von \ac{AWS} sollen neue Abteilungen berücksichtigt werden.
Dafür müssen diese dem bestehenden \ac{ECR} und \ac{ECN} hinzugefügt werden.

Bei der Erstellung der \acsp{ECR} werden zudem Startparameter abgefragt.
Das sind Boolean-Attribute, die die automatische Auswahl von Abteilungen beeinflussen.
Dies ist eine Erleichterung für den Change Admin, da dann nicht jedes Mal alle Abteilungen manuell ausgefüllt werden müssen.
Der \ac{ECR} ist für alle Organisationen gleich.
Allerdings benötigen nicht alle Organisationen alle Attribute, die auf dem \ac{ECR} definiert wurden.
Dafür wurde von einem anderen Softwareunternehmen eine XML-Datei entwickelt, die nicht benötigte Attribute von einer Organisation vom UI versteckt.
Es sollen neue Boolean-Attribute definiert werden, die nur für die \ac{AWS} Organisationen sichtbar sind.
Dazu müssen diese dem \ac{ECR} hinzugefügt werden und die in der XML-Datei für die anderen Organisationen versteckt werden.

Um den Change Admin bei der Auswahl der beteiligten Abteilungen zu unterstützen, wurde von einem anderen Softwareunternehmen eine benutzerdefinierte JSP Seite entwickelt, die die Abteilungen in einer Liste darstellt.
Der Change Admin kann dann über Checkboxen Abteilungen für den Prozess entfernen oder hinzufügen.
Dieses Pop-Up-Fenster wurde von einem anderen Softwareunternehmen in einem bereits abgeschlossenem Projekt entwickelt.


Zudem werden Abteilungen aktuell in einem eigens dafür angefertigten Pop-Up-Fenster ausgewählt.
Hier kann der Change Admin über Checkboxen Abteilungen auswählen, die am Change Prozess beteiligt sein sollen.
Dieses Pop-Up-Fenster wurde von einem anderen Softwareunternehmen in einem bereits abgeschlossenem Projekt entwickelt.
Allerdings bringt diese Lösung einige Probleme mit sich.
Zum einen verzögern Customizations, die von den OOTB-Windchill-Konfigurationen abweichen, oft größere Windchill Upgrades, da nicht garantiert ist, dass die Customization auch in zukünftigen Versionen funktioniert.
Zum anderen können nur Administratoren mit Zugriff zur virtuellen Maschine Änderungen an der Struktur des Pop-Up-Fensters vornehmen (\zB hinzufügen oder entfernen von Abteilungen), da die Customization als JSP-Datei auf dem Server liegt.
Um diese beiden Probleme zu beseitigen, soll hier auf eine Lösung zurückgegriffen werden, die bereits von Haus aus in Windchill implementiert ist, nämlich den 'Set Up Participants'-Tab.
Da es sich um eine OOTB-Funktion von Windchill handelt, ist es unwahrscheinlich, dass es damit Probleme bei einem Upgrade gibt.
Außerdem können Windchill-Administratoren über den Workflow Template Editor Änderungen an den Abteilungen vornehmen, ohne Zugriff auf die VM zu benötigen.

Zudem sind die aktuellen Workflowprozesse sehr unstrukturiert dargestellt und sind kaum dokumentiert.
Dies macht es schwieriger für Entwickler, die nicht mit den Prozessen vertraut sind, Erweiterungen oder Änderungen vorzunehmen.
\subsection{Wirtschaftlichkeitsanalyse} %todo: finish section
\label{sec:Wirtschaftlichkeitsanalyse}

Da die exakten Gesamtkosten des Projekts nicht offenbart wurden, gehen wir von einem empfohlenen Festpreis von 50.000 € aus, wie mir mitgeteilt wurde. %TODO: Silvia fragen, wegen Projektkosten
Der Anteil dieser spezifischen Projektarbeit am Gesamtprojekt wird auf 80 \% geschätzt.
Daher kalkulieren wir einen Umsatz von 40.000 € für die Umsetzung gemäß der hier beschriebenen Aufgabenstellung.

\subsubsection{Projektkosten}
\subsubsection*{Ausbildungsvergütung}
Hier haben wir ein Bruttoentgelt von 1080 €.

\subsubsection*{Arbeitgeberanteil Sozialversicherung}
Die Arbeitgeber-Sozialabgaben betragen 199,92 € ohne die Unfallversicherung.
Für die Unfallversicherung liegen mir keine Zahlen vor, auch eine Internetrecherche half nicht weiter.
Deswegen wurde mir von der Personalabteilung geraten mit 20 € pro Monat zu rechnen.

\subsubsection*{Abstimmungskosten}
Der Projektleiter von Arsandis führte mehrere interne Abstimmungsgespräche mit mir zum Fortschritt des Projekts durch.
Diese Gespräche hatten das Ziel sicherzustellen, dass die Qualitätsanforderungen erfüllt werden und die Umsetzung auf bewährten Methoden basiert, während die Lösung sich gut in den Gesamtentwurf des Kundensystems einfügt.
Zusätzlich nahm der Projektleiter, wie in anderen Teilen des Projekts auch, an Meetings mit dem Kunden teil.
Konkrete Zahlen dafür sind für mich nicht einsehbar.
Es wurde mir geraten, mit einem festen Stundensatz von 100 € zu kalkulieren.

\subsubsection*{Gemeinkosten}
Intern verwenden wir keinen Gemeinkostenfaktor.
Ungeachtet dessen müssen wir jedoch sicherstellen, dass unsere Gemeinkosten gedeckt werden.
Aus diesem Grund werde ich eine weitere Annahme treffen, um eine standardmäßige Wirtschaftlichkeitsanalyse durchzuführen, da mir die genauen Zahlen nicht zugänglich sind.
\begin{itemize}
	\item Energiekosten: Heizung, Strom = 15 \%
	\item Mietkosten: Büroräume = 15 \%
	\item Ausrüstungskosten: Clients, Monitore, Server, Peripherie, Lizenzen = 10 \%
	\item Gehälter für Marketingabteilung: 10 \%
	\item Gehälter für Personalabteilung: 10 \%
\end{itemize}

Das führt uns zu einem Gemeinkostenzuschlag von 60 \%

\subsubsection*{Gesamtkosten}
Hier werden die Kosten eines Arbeitstages durch das Herunterbrechen des Jahresverdiensts auf einen Tag ermittelt.
\citet{ArbeitstageImJahr}{Für das Jahr 2023 wurden 251 Arbeitstage angenommen}.

Kosten für Arbeitstage des Prüflings = 12960 € / (251 Arbeitstage - 24 Urlaubstage) * 10 = 570 €

Kosten für Abstimmungsgespräche = Stundensatz * Anzahl Stunden = 100 € * 6 = 600 €

Gesamtkosten = (Kosten für Abstimmungsgespräche + Kosten für Arbeitstage des Prüflings) * Gemeinkostenzuschlag = (570 € + 600 €) * 1,60 = 1872 €

\subsubsection{Amortisation} %todo: finish section


\subsection{Lastenheft/Fachkonzept}
\label{sec:Lastenheft}
Zu Beginn des Projekts hat der Kunde bereits einen Entwurf des Lastenhefts ausgearbeitet, welcher zusammen mit dem Autor und dem Projektleiter von Arsandis weiter angepasst wurde.
Dort sind alle Anforderungen des Auftraggebers für die Anpassungen der aktuellen Anwendung enthalten.
Im \Anhang{app:Lastenheft} befindet sich ein Auszug aus dem Lastenheft.
