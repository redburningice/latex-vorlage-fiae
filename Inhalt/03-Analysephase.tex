% !TEX root = ../Projektdokumentation.tex
\section{Analysephase} 
\label{sec:Analysephase}


\subsection{Ist-Analyse} 
\label{sec:IstAnalyse}
\begin{itemize}
	\item Wie ist die bisherige Situation (\zB bestehende Programme, Wünsche der Mitarbeiter)?
	\item Was gilt es zu erstellen/verbessern?
\end{itemize}

Der \ac{ECM} besteht aus den zwei Komponenten \ac{ECR} und \ac{ECN}.
Aktuell werden Aufgaben im \ac{ECN} getrackt, dies ist so von Windchill eigentlich nicht vorgesehen.
In Zukunft sollen Aufgaben deswegen in einem \ac{ECA} abgebildet werden.
Dies führt dazu, dass die konkreten Implementierungsaufgaben logisch von der Planungsphase (ECN) abgegrenzt wird, was es dem Change Admin einfacher macht, einen Überblick über die aktuelle Situation zu bekommen.

Zudem werden Abteilungen aktuell in einem eigens dafür angefertigten Pop-Up-Fenster ausgewählt.
Hier kann der Change Admin über Checkboxen Abteilungen auswählen, die am Change Prozess beteiligt sein sollen.
Dieses Pop-Up-Fenster wurde von einem anderen Softwareunternehmen in einem bereits abgeschlossenem Projekt entwickelt.
Allerdings bringt diese Lösung einige Probleme mit sich.
Zum einen verzögern Customizations oft größere Windchill Upgrades, da nicht garantiert ist, dass die Customization auch in zukünfigen Versionen funktioniert.
Zum anderen können nur Administratoren mit Zugriff zur virtuellen Maschine Änderungen an der Struktur des Pop-Up-Fensters vornehmen (\zB hinzufügen oder entfernen von Abteilungen), da die Customization als JSP-Datei auf dem Server liegt.
Um diese beiden Probleme zu beseitigen, soll hier auf eine Lösung zurückgegriffen werden, die bereits von Haus aus in Windchill implementiert ist, nämlich den 'Set Up Participants'-Tab.
Da es sich um eine OOTB-Funktion von Windchill handelt, ist es unwahrscheinlich, dass es damit Probleme bei einem Upgrade gibt.
Außerdem können Windchill-Administratoren über den Workflow Template Editor Änderungen an den Abteilungen vornehmen, ohne Zugriff auf die VM zu benötigen.

\subsection{Wirtschaftlichkeitsanalyse}
\label{sec:Wirtschaftlichkeitsanalyse}
\subsubsection{Projektkosten}

\subsubsection{Amortisation}

\subsection{Anwendungsfälle}
\label{sec:Anwendungsfaelle}
\begin{itemize}
	\item Welche Anwendungsfälle soll das Projekt abdecken?
	\item Einer oder mehrere interessante (!) Anwendungsfälle könnten exemplarisch durch ein Aktivitätsdiagramm oder eine EPK detailliert beschrieben werden.
\end{itemize}

\paragraph{Beispiel}
Ein Beispiel für ein Use Case-Diagramm findet sich im \Anhang{app:UseCase}.


\subsection{Qualitätsanforderungen}
\label{sec:Qualitaetsanforderungen}
\begin{itemize}
	\item Welche Qualitätsanforderungen werden an die Anwendung gestellt (\zB hinsichtlich Performance, Usability, Effizienz \etc )?
\end{itemize}


\subsection{Lastenheft/Fachkonzept}
\label{sec:Lastenheft}
\begin{itemize}
	\item Auszüge aus dem Lastenheft/Fachkonzept, wenn es im Rahmen des Projekts erstellt wurde.
	\item Mögliche Inhalte: Funktionen des Programms (Muss/Soll/Wunsch), User Stories, Benutzerrollen
\end{itemize}

\paragraph{Beispiel}
Ein Beispiel für ein Lastenheft findet sich im \Anhang{app:Lastenheft}. 
