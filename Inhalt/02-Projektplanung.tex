% !TEX root = ../Projektdokumentation.tex
\section{Projektplanung} 
\label{sec:Projektplanung}


\subsection{Projektphasen}
\label{sec:Projektphasen}

Für die Umsetzung des Projekts wurden 76 Stunden angesetzt.
Der Start des Projekts erfolgt am 07.11.2023 und bis zum 27.11.2023 wird es abgeschlossen sein.
Der Tabelle \ref{tab:Zeitplanung} kann entnommen werden, in welche Hauptphasen das Projekt gegliedert wurde.
\tabelle{Zeitplanung}{tab:Zeitplanung}{ZeitplanungKurz}\\
Eine detailliertere Zeitplanung findet sich im \Anhang{app:Zeitplanung}.
Im Vergleich zum Projektantrag wurde die Erstellung der Projektdokumentation noch ergänzt.
Dadurch erhöht sich die Gesamtdauer des Projekts auf 80 Stunden und die für die Erstellung eines Benutzerhandbuchs geschätzte Zeit wurde auf 1 Stunde reduziert.


\subsection{Ressourcenplanung}
\label{sec:Ressourcenplanung}

Für die Durchführung des Projekts werden folgende Ressourcen verwendet.

\subsubsection{Hardware}
Von meinem Büroarbeitsplatz aus wird mein Windows 10 Arbeitslaptop genutzt.

\subsubsection{Entwicklungsumgebung}
Im Vorfeld des Projekts wurde bereits eine virtuelle Maschine mit dem Betriebssystem Windows Server 2019 aufgesetzt.
Diese ist mit 16 GB RAM ausgestattet.
Auf dieser wurde Windchill in der Version 11.1 M020-CPS026 installiert.
Windchill benötigt eine Oracle Datenbank der Version 19C und eine Java Runtime Environment der Version 1.8.0 Update 202.
Für die Erstellung des Codes wird die IDE Eclipse auf der \acs{VM} verwendet.
Die User des Windchillsystems werden im Open Source LDAP Programm OpenDJ 3.0.0 verwaltet.
Eine weitere Komponente, die benötigt wird, ist ein Apache 2.4 Webserver.

\subsubsection{Testsystem}
Das Testsystem läuft auf einem Linux-Server ohne User Interface.
Für die Verbindung zur Weboberfläche und zum Linux-Server wird das VPN Programm GlobalProtect genutzt.

\subsubsection{Weitere Software}
Zur Erstellung der Projektdokumentation wird IntelliJ mit dem TeXiFy Plugin und die MiKTeX Distribution für LaTeX verwendet.

\subsubsection{Personal}
Außerdem haben mich folgende Personen bei meinem Projekt unterstützt.
Der Product Owner von \acs{AWS} legt die Anforderungen fest und nimmt das Projekt ab.
Ein Experte für das Windchill-System von \acs{AWS} hilft mir dabei, mich in die Customizations von \acs{AWS} einzuarbeiten.
Als Entwickler führe ich die Umsetzung des Projektes durch.
Der Projektleiter von Arsandis überprüft die Umsetzung und den Code.
Eine Windchill-Anwendungsspezialistin von Arsandis unterstützt den mich beim Implementieren und Testen.
Eine Mitarbeiterin der Personalabteilung hilft beim Aufstellen der Wirtschaftlichkeitsanalyse.

\subsection{Entwicklungsprozess}
\label{sec:Entwicklungsprozess}

Die Umsetzung meines Projekts wird grundsätzlich agil ablaufen, in dem der aktuelle Status und Änderungen an den Anforderungen nach Bedarf mit den Kunden diskutiert werden.
Außerdem gibt es regelmäßige Besprechungen mit dem Projektleiter, um den Status des Projekts zu besprechen.
Dort wird auch diskutiert, ob es noch Optimierungsbedarf bei der Implementierung gibt.
