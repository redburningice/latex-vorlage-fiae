% !TEX root = ../Projektdokumentation.tex
\section{Entwurfsphase} 
\label{sec:Entwurfsphase}

\subsection{Zielplattform}
\label{sec:Zielplattform}

Bei der Zielplattform handelt es sich um PTC Windchill PDMLink, eine web-basierte Anwendung zur Verwaltung von Unternehmensdaten.
Windchill ist mit jedem modernen Browser kompatibel.
Windchill besteht aus den folgenden Grundbestandteilen:

\subsubsection{Windchill PDMLink}
Windchill und Windchill PDMLink werden oft synonym verwendet, da PDMLink das Basispaket von Windchill darstellt.
Windchill ist eine \ac{PLM} Webanwendung von PTC, die das Verwalten von Unternehmensobjekten für die Herstellungsindustrie vereinfacht.
Besonders Wert wird dabei auf die Verwaltung von CAD-Modellen gelegt.
CAD-Modelle sind am Computer gefertigte zwei- und dreidimensionale Zeichnungen von Konstruktionsobjekten.
Mit dem Windchill Workgroup Manager existiert eine Schnittstelle zwischen Windchill und vielen CAD-Anwendungen, die auf dem Markt existieren (wie \zB Creo Parametric, CATIA oder Solidworks).
Das Hauptgeschäftsfeld von PTC ist das Engineering, darunter vor allem die Automobilindustrie oder die Luft- und Raumfahrtindustrie.
Allerdings bietet PTC noch eine große Zahl an Erweiterungen an, um \zB auch die Textilindustrie optimal in Windchill integrieren zu können.

\subsubsection{Programmiersprache}
Da Windchill eine Java-Applikation ist, werden tiefer gehende Anpassungen durch Java-Code erfolgen.
Auch der Code in den Workflow-Templates wird in Java definiert.
Um die Benutzeroberfläche um neue Elemente oder Fenster zu erweitern wird auch das JSP Framework genutzt.

\subsubsection{Datenbank}
Unser Entwicklungsserver läuft auf einer Oracle 19C Datenbank, Windchill ist allerdings auch mit den Datenbanken von Microsoft kompatibel.

\subsection{Geschäftslogik} %todo: finish section
\label{sec:Geschaeftslogik}

\begin{itemize}
	\item Modellierung und Beschreibung der wichtigsten (!) Bereiche der Geschäftslogik (\zB mit Kom\-po\-nen\-ten-, Klassen-, Sequenz-, Datenflussdiagramm, Programmablaufplan, Struktogramm, EPK).
	\item Wie wird die erstellte Anwendung in den Arbeitsfluss des Unternehmens integriert?
\end{itemize}


\subsection{Maßnahmen zur Qualitätssicherung} %todo: finish section
Das Testing erfolgt manuell durch \glqq durchklicken\grqq{} der Workflowprozesse.
Dabei erfolgen mehrere Arsandis-Interne Tests auf unserer Entwicklungsumgebung, bei dem eine Kollegin den Prüfling unterstützt.
Hier wird überprüft, ob die korrekte Funktionsweise der Anpassungen gegeben ist.
Dafür wurde ein einfacher Testplan erstellt. %todo: add testplan to anhang

\subsection{Pflichtenheft/Datenverarbeitungskonzept} %todo: finish section, haben wir ein Pflichtenheft? GGF entfernen
\label{sec:Pflichtenheft}
\begin{itemize}
	\item Auszüge aus dem Pflichtenheft/Datenverarbeitungskonzept, wenn es im Rahmen des Projekts erstellt wurde.
\end{itemize}

\paragraph{Beispiel}
Ein Beispiel für das auf dem Lastenheft (siehe Kapitel~\ref{sec:Lastenheft}: \nameref{sec:Lastenheft}) aufbauende Pflichtenheft ist im \Anhang{app:Pflichtenheft} zu finden.
