% !TEX root = ../Projektdokumentation.tex
\section{Entwurfsphase} 
\label{sec:Entwurfsphase}

\subsection{Zielplattform}
\label{sec:Zielplattform}

Bei der Zielplattform handelt es sich um PTC Windchill PDMLink, eine web-basierte Anwendung zur Verwaltung von Unternehmensdaten.
Windchill ist mit jedem modernen Browser kompatibel.

\subsubsection{Windchill PDMLink}
Windchill PDMLink ist das Basispaket von Windchill.
Windchill ist eine \ac{PLM} Webanwendung von PTC, die das Verwalten von Unternehmensobjekten für die Herstellungsindustrie vereinfacht.
Im PDMLink Paket wird besonders Wert auf die Verwaltung von CAD-Modellen gelegt.
CAD-Modelle sind am Computer gefertigte zwei- und dreidimensionale Zeichnungen von Konstruktionsobjekten.
Mit dem Windchill Workgroup Manager existiert eine Schnittstelle zwischen Windchill und vielen CAD-Anwendungen, die auf dem Markt existieren (wie \zB Creo Parametric, CATIA oder Solidworks).
Das Hauptgeschäftsfeld von PTC ist das Engineering, darunter \zB die Automobilindustrie oder die Luft- und Raumfahrtindustrie.
Allerdings bietet PTC noch eine große Zahl an Erweiterungen an, um \zB auch die Textilindustrie optimal in Windchill integrieren zu können.

\subsubsection{Programmiersprache}
Da Windchill eine Java-Applikation ist, werden tiefer gehende Anpassungen durch Java-Code erfolgen.
Auch der Code in den Workflow-Templates wird in Java definiert.
Um die Benutzeroberfläche um neue Elemente oder Fenster zu erweitern wird das JSP Framework genutzt.

\subsubsection{Datenbank}
Der Entwicklungsserver läuft auf einer Oracle 19C Datenbank.

\subsection{Geschäftslogik}
\label{sec:Geschaeftslogik}

\subsubsection{ECA}
Im \Anhang{Aktivitaetsdiagramm} befindet sich das Aktivitätsdiagramm für den ECA-E Prozess.
Das ist eine spezielle Form des ECA-Prozesses, der alle Aufgaben der Ingenieure enthält.
Der Gegenpart dazu ist der ECA-NE, in dem sich alle Aufgaben für die Nicht-Ingenieure sammeln.
Dieser ist ähnlich aufgebaut wie der ECA-E, deswegen wird hier nur auf einen Prozess näher eingegangen.

Zuerst werden die \glqq Affected Objects\grqq{}, also die Objekte, die im Änderungsprozess geändert werden sollen, vom ECR zum ECA-E kopiert.
Dadurch kann vom ECA-E auf die Objekte zugegriffen werden und es können \glqq Resulting Objects\grqq{} erzeugt werden.
Mit dieser speziellen ECA Funktion kann der Ingenieur festlegen, welchen Lebenszyklusstatus das Objekt nach der Fertigstellung des Change-Prozesses erhält.

Als nächstes wird das \glqq Product Line Attribute\grqq{} eingelesen.
Damit wird bestimmt welches Team Template im nächsten Schritt geladen wird.
Es gibt für jede Produktlinie jeweils zwei Team Templates.
Team Templates stellen ein von den Administratoren vordefiniertes Mapping zwischen Usern und Rollen dar.
Dies erlaubt es uns eine Standardkonfiguration in den Workflow zu laden.
Damit müssen Change Administratoren nicht alle User händisch zuweisen.

Danach werden die Startparameter des ECR in den ECA-E geladen.
Wie beim ECR beeinflussen diese auch hier die automatische Rollenauswahl.
Wurde also \zB beim Erstellen des ECR das Attribut \glqq Drawings\grqq{} auf Nein gesetzt, so wird der User dieser Rolle automatisch im nächsten Schritt aus dem ECA-E Prozess entfernt.

Möchte der Change Admin von der automatischen Vorauswahl abweichen, so kann er dies in der anschließenden Aufgabe \glqq Select users\grqq{} machen.

Im Anschluss daran werden die ausgewählten Ingenieure dazu aufgefordert ihre Aufgabe abzuschließen.
In dieser können sie Änderungen an den Affected Objects vornehmen.

Haben alle Ingenieure ihre Aufgaben abgeschlossen, werden die Affected Objects automatisch vom System gesperrt.
Dadurch können während der Review-Phase des Change Admin keine Änderungen mehr an den Change-Objekten vorgenommen werden.

Abschließend kann der Change Admin nun entscheiden, ob die Aufgaben zufriedenstellend abgeschlossen wurden.
Ist er mit der Arbeit einer Abteilung nicht zufrieden, so kann er diesen eine \glqq Rework\grqq{}-Task zukommen lassen.

Wenn der Change Admin zufrieden ist, so ist der ECA-E abgeschlossen und es wird mit dem ECA-NE fortgefahren, in dem der gleiche Prozess, nur mit Nicht-Ingenieuren, durchlaufen wird.


\subsection{Maßnahmen zur Qualitätssicherung}
Das Testing erfolgt manuell durch \glqq durchklicken\grqq{} der Workflowprozesse.
Dabei erfolgen mehrere Arsandis-Interne Tests auf unserer Entwicklungsumgebung, bei dem eine Kollegin den Prüfling unterstützt.
Hier wird überprüft, ob die korrekte Funktionsweise der Anpassungen gegeben ist.