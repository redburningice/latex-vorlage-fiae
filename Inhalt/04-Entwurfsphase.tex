% !TEX root = ../Projektdokumentation.tex
\section{Entwurfsphase} 
\label{sec:Entwurfsphase}

\subsection{Zielplattform} %todo: finish section
\label{sec:Zielplattform}

\begin{itemize}
	\item Beschreibung der Kriterien zur Auswahl der Zielplattform (\ua Programmiersprache, Datenbank, Client/Server, Hardware).
\end{itemize}

\subsection{Geschäftslogik} %todo: finish section
\label{sec:Geschaeftslogik}

\begin{itemize}
	\item Modellierung und Beschreibung der wichtigsten (!) Bereiche der Geschäftslogik (\zB mit Kom\-po\-nen\-ten-, Klassen-, Sequenz-, Datenflussdiagramm, Programmablaufplan, Struktogramm, EPK).
	\item Wie wird die erstellte Anwendung in den Arbeitsfluss des Unternehmens integriert?
\end{itemize}


\subsection{Maßnahmen zur Qualitätssicherung} %todo: finish section
\label{sec:Qualitaetssicherung}
\begin{itemize}
	\item Welche Maßnahmen werden ergriffen, um die Qualität des Projektergebnisses (siehe Kapitel~\ref{sec:Qualitaetsanforderungen}: \nameref{sec:Qualitaetsanforderungen}) zu sichern (\zB automatische Tests, Anwendertests)?
	\item \Ggfs Definition von Testfällen und deren Durchführung (durch Programme/Benutzer).
\end{itemize}

Das Testing erfolgt manuell durch \glqq durchklicken\grqq{} der Workflowprozesse.
Dabei erfolgen mehrere Arsandis-Interne Tests auf unserer Entwicklungsumgebung, bei dem eine Kollegin den Prüfling unterstützt.


Sobald wir unseren Testprozess erfolgreich abgeschlossen haben, werden die Anpassungen auf das Testsystem des Kunden übertragen.
Daran anschließend startet die zweite Testphase, die von einigen Mitarbeitern von \ac{AWS} durchgeführt werden.
Nach ihren Tests erhalten wir ein Testprotokoll, in dem alle Fehler beziehungsweise Auffälligkeiten dokumentiert werden.

\subsection{Pflichtenheft/Datenverarbeitungskonzept} %todo: finish section
\label{sec:Pflichtenheft}
\begin{itemize}
	\item Auszüge aus dem Pflichtenheft/Datenverarbeitungskonzept, wenn es im Rahmen des Projekts erstellt wurde.
\end{itemize}

\paragraph{Beispiel}
Ein Beispiel für das auf dem Lastenheft (siehe Kapitel~\ref{sec:Lastenheft}: \nameref{sec:Lastenheft}) aufbauende Pflichtenheft ist im \Anhang{app:Pflichtenheft} zu finden.
