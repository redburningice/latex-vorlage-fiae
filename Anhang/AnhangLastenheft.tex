\subsection{Lastenheft (Auszug)}
\label{app:Lastenheft}
Es folgt ein Auszug aus dem Lastenheft mit Fokus auf die Anforderungen:

Die Anwendung muss folgende Anforderungen erfüllen: 
\begin{enumerate}[itemsep=0em,partopsep=0em,parsep=0em,topsep=0em]
\item Verarbeitung der Moduldaten
	\begin{enumerate}
	\item Die Anwendung muss die von Subversion und einem externen Programm bereitgestellten Informationen (z.B. Source-Benutzer, -Datum, Hash) verarbeiten.
	\item Auslesen der Beschreibung und der Stichwörter aus dem Sourcecode.
	\end{enumerate}
\item Darstellung der Daten
	\begin{enumerate}
	\item Die Anwendung muss eine Liste aller Module erzeugen inkl. Source-Benutzer und -Datum, letztem Commit-Benutzer und -Datum für alle drei Umgebungen. 
	\item Verknüpfen der Module mit externen Tools wie z.B. Wiki-Einträgen zu den Modulen oder dem Sourcecode in Subversion.
	\item Die Sourcen der Umgebungen müssen verglichen und eine schnelle Übersicht zur Einhaltung des allgemeinen Entwicklungsprozesses gegeben werden. 
	\item Dieser Vergleich muss auf die von einem bestimmten Benutzer bearbeiteten Module eingeschränkt werden können. 
	\item Die Anwendung muss in dieser Liste auch Module anzeigen, die nach einer Bearbeitung durch den gesuchten Benutzer durch jemand anderen bearbeitet wurden. 
	\item Abweichungen sollen kenntlich gemacht werden. 
	\item Anzeigen einer Übersichtsseite für ein Modul mit allen relevanten Informationen zu diesem.
	\end{enumerate}
\item Sonstige Anforderungen
	\begin{enumerate}
	\item Die Anwendung muss ohne das Installieren einer zusätzlichen Software über einen Webbrowser im Intranet erreichbar sein.
	\item Die Daten der Anwendung müssen jede Nacht \bzw nach jedem SVN-Commit automatisch aktualisiert werden.
	\item Es muss ermittelt werden, ob Änderungen auf der Produktionsumgebung vorgenommen wurden, die nicht von einer anderen Umgebung kopiert wurden. Diese Modulliste soll als Mahnung per E-Mail an alle Entwickler geschickt werden (Peer Pressure).
	\item Die Anwendung soll jederzeit erreichbar sein.
	\item Da sich die Entwickler auf die Anwendung verlassen, muss diese korrekte Daten liefern und darf keinen Interpretationsspielraum lassen.
	\item Die Anwendung muss so flexibel sein, dass sie bei Änderungen im Entwicklungsprozess einfach angepasst werden kann.
	\end{enumerate}
\end{enumerate}

